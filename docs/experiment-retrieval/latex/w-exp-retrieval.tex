\documentclass[a4paper,12pt]{article}
%\documentclass[a4paper,10pt]{scrartcl}

\usepackage[utf8x]{inputenc}
\usepackage{amsfonts}
\usepackage{amsmath,esint}
\usepackage{graphicx}
\usepackage{pdfpages}
\usepackage{sansmath}
\usepackage{hyperref}
\usepackage{natbib}
\usepackage{caption}

\usepackage{tikz}


\definecolor{boiseBlue} {RGB}{29,72,159}
\definecolor{rojoAmor} {RGB}{171,13,4}
\definecolor{moradoAmor} {RGB}{93,8,113}
\definecolor{verdeAmor} {RGB}{98,158,31}
\definecolor{negro} {RGB}{10,10,10}
\definecolor{lgreen} {RGB}{180,210,100}
\definecolor{dblue}  {RGB}{20,66,129}
\definecolor{ddblue} {RGB}{11,36,69}
\definecolor{lred}   {RGB}{220,0,0}
\definecolor{nred}   {RGB}{224,0,0}
\definecolor{norange}{RGB}{230,120,20}
\definecolor{nyellow}{RGB}{255,221,0}
\definecolor{ngreen} {RGB}{98,158,31}
\definecolor{dgreen} {RGB}{78,138,21}
\definecolor{nblue}  {RGB}{28,130,185}
\definecolor{jblue}  {RGB}{20,50,100}

\usepackage{listings}
\usepackage{xcolor}
\usepackage{verbatim}
\lstset{language=C++,
		basicstyle=\ttfamily,
	       backgroundcolor=\color{black!5}\ttfamily,
                keywordstyle=\color{nblue}\ttfamily,
                stringstyle=\color{nred}\ttfamily,
                commentstyle=\color{ngreen}\ttfamily,
                morecomment=[l][\color{moradoAmor}]{\#}
}

\newenvironment{rcases}{\left.\begin{aligned}}{\end{aligned}\right\rbrace}

\usepackage{titling}
\setlength{\droptitle}{-6em}   % This is your set screw

\renewcommand{\familydefault}{\sfdefault}

\newcommand{\specialcell}[2][c]{%
  \begin{tabular}[#1]{@{}c@{}}#2\end{tabular}}
% \specialcell{Foo\\bar}

\title{Wave experiment\\{\normalsize Diego Domenzain}}
\author{}
\date{}

\pdfinfo{%
  /Title    ()
  /Author   ()
  /Creator  ()
  /Producer ()
  /Subject  ()
  /Keywords ()
}

\begin{document}
\maketitle
%-------------------
% before
%-------------------
\section*{Constants and units}
\begin{table}[!h]
\centering
\begin{tabular}{ l  c | l }
$\varepsilon_o = 8.85\cdot10^{-12}$ &  F/m & vacuum permittivity \\
$\mu_o = 4\pi\cdot 10^{-7} $ & H/m & vacuum permeability \\
$c=\frac{1}{\sqrt{\varepsilon_o\mu_o}}=299792458$ & m/s & light speed in vacuum \\
\end{tabular}
\caption{ Constants.}
 \label{tbl:ctes}
\end{table}
%
\begin{table}[!h]
\centering
\begin{tabular}{ r c l | l }
ns & $\to$ & s & $\cdot 10^{-9}$ \\
MHz & $\to$ & Hz & $\cdot 10^{6}$ \\
GHz & $\to$ & Hz & $\cdot 10^{9}$ \\
MHz & $\to$ & GHz & $\cdot 10^{-3}$ \\
\end{tabular}
\caption{ Unit conversions.}
 \label{tbl:cnvs}
\end{table}
%-------------------
% before
%-------------------
\section*{Before going to the field}
\begin{itemize}
\item Choose $f_o$ and maximum target permittivity $\varepsilon_{max}$.
\item Compute $\lambda_o$,
 \begin{align}
\lambda_o &= \frac{c}{f_o\sqrt{\varepsilon_{max}}},\\
 &\approx \frac{300}{f_{o,MHz}\sqrt{\varepsilon_{max}}} \;\;\;\; [1/m].
\end{align}
\item Choose $\Delta sr \geq \lambda_o$ and compute $\Delta r \leq \lambda_o/2$. For example, 
\begin{align*}
\Delta sr=\lambda_o+\text{odometer-wheel}
\hspace{2em} \text{and} \hspace{2em}
\Delta r = \lambda_o/4.
\end{align*}
\end{itemize}
%
\begin{table}[!h]
\centering
\begin{tabular}{ r | l | l}
$f_o$ MHz & $\lambda_o$ m & $\lambda_o/4$ m \\
\hline
50 & 1 & 0.2 \\
100 & 0.5 & 0.1 \\
200 & 0.25 & 0.05 \\
250 & 0.2 & 0.05 \\
500 & 0.1 & 0.02 \\
\hline
\end{tabular}
\caption{Approximate wavelengths (and spacings) for $\varepsilon_{max}$ = 30. You can do your own by computing $\lambda_o=300/f_{o,MHz}/\sqrt{\varepsilon_{max}}$.}
 \label{tbl:wavelength}
\end{table}
%-------------------
% after
%-------------------
\section*{After going to the field}
\subsection*{Save data and time-shift: \texttt{ss2gerjoii\_w.m}}
\begin{enumerate}
\item Loop through done survey lines and save all of the data to \texttt{.mat} in disk, and store $r$ and $t$ as $r_{all}$ and $t_{all}$ in memory.
\item Loop over all saved lines and rewrite them,
\begin{enumerate}
\item get $r$ and $t$ for line,
\item correct for time-shift: $t\gets t-t_{all}(t=t_o)$, $d^o\gets d^o$ corrected,
\item transform $\{r,t,\Delta t,\Delta r,\Delta sr, f_o\}$ to $m,ns$ and $GHz$,
\item use $\Delta s$, $\Delta sr$ and $r_{all}$ to create source-receivers tuples $\{s,r\}$ in real coordinates.
\end{enumerate}
\item Store all $\{s,r\}$ to disk.
\end{enumerate}
%
\subsection*{Quick-see and compute $\Delta t,\Delta x$: \texttt{datavis\_w.m}}
\begin{enumerate}
\item Choose data $d^o$ together with $\{r,t,\Delta t,\Delta r,\Delta sr, f_o\}$.
\item Dewow.
\item See $d^o$ in the frequency domain and filter out unwanted signal with a bandpass filter $[f_{min},\, f_{max}]$.
\item Find first arrival event time $t_{fa}$ before which all recordings should be silent. For example, $t_{fa}=\Delta sr/c$.
\item Choose up to which receivers the data looks good.
\item Choose $v_{min}$ and compute $\varepsilon_{max} = (c/v_{min})^2$ to see if $v_{min}$ makes sense.
\item Compute $\Delta x$ using the {\it wavelength condition},
\begin{align}
\Delta x &= \frac{v_{min}}{n_\lambda\cdot f_{max}}.
\end{align}
\item Choose $\varepsilon_{min}\geq1$ and compute $\Delta t_\bullet$ with the {\it cfl condition},
\begin{align}
v_{max} &= \frac{c}{\sqrt{\varepsilon_{min}}},\\
 \Delta t_\bullet &= c_f \cdot \frac{1}{v_{max}\sqrt{1/\Delta x^2 + 1/\Delta z^2}}.
\end{align}
Most likely, $\Delta t_\bullet<\Delta t$ because the gpr system computes $\Delta t$ to satisfy the {\it Nyquist condition} on the frequency it can measure ($\approx 1.2\,GHz$). 
\end{enumerate}
%
%
\subsection*{Get the data ready for the fwi scheme: \texttt{data2fwi\_w.m}}
\begin{enumerate}
\item Choose data $d^o$ together with $\{r,t,\Delta t,\Delta r,\Delta sr, f_o\}$.
\item Dewow $d^o$.
\item Bandpass filter $d^o$ with $[f_{min},\, f_{max}]$.
\item Remove unwanted receivers.
\item Perform 2.5d$\to$2d conversion.
\item Mute unwanted events. For example, everything before $t_{fa}$.
\item Interpolate $d^o$ on time vector $t_\bullet=t(1):\Delta t_\bullet:t(n_t)$ and set,
\begin{align}
t &\gets t_\bullet \\
 \Delta t &\gets \Delta t_\bullet.
\end{align}
The {\it spline} interpolation works best.
\item Estimate source wavelet.
\item Save $d^o$ to disk with its respective $\{s,r\}$ on binned coordinates in the new discretized vectors $x$ and $z$.
\item Loop until all data has been preprocessed.
\end{enumerate}
%
\subsection*{Get parameters ready for fwi scheme: \texttt{param2fwi\_w.m}}
\begin{itemize}
\item Complete structures,
\begin{itemize}
\item[$\circ$] \texttt{parame\_}
\item[$\circ$] \texttt{geome\_}
\item[$\circ$] \texttt{finite\_}
\end{itemize}
with the space and time parameters computed in \texttt{data2fwi\_w.m}.
\end{itemize}
%
\subsection*{Complete \texttt{gerjoii\_} structure for inversion: \texttt{w\_load.m}}
For one \texttt{radargram\_} structure,
\begin{itemize}
\item Load source location $[i_z, i_x]$ a $1\times 2$ vector.
\item Load receiver locations $[i_x, i_z]$ an $n_r\times 2$ matrix.
\item Modify source and receivers to account for PML and air.
\item Build $M_w$.
\item Load $d^o$ (of size $n_t\times n_r$) and $\operatorname{std}(d^o)$ (of size $n_t n_r\times 1$).
\item Load source wavelet.
\end{itemize}
The source wavelet is updated and saved to disk in the \texttt{wvlets\_} structure after each iteration.
%
\subsection*{Inversion routine}
\begin{enumerate}
\item Begin while loop.
\item Compute update,
\begin{itemize}
\item Begin for loop through experiments,
\item load source-receivers, $d^o$, $\operatorname{std}(d^o)$ and wavelet,
\item build $M_w$,
\item forward run,
\item evaluate objective function,
\item compute gradient
\item compute step-size,
\item compute and store update,
\item estimate source wavelet and store,
\item end for loop.
\end{itemize}
Stack all updates into one update.
\item Update.
\item End while loop.
\end{enumerate}
%
\subsection*{Extracting more information}
\begin{enumerate}
\item Build source-receiver matrix of size $n_s\times n_r$ on binned coordinates in discretized vectors $x$ and $z$.
\item Interferometry.
\item Velocity semblance.
\item Source estimation. 
\end{enumerate}
%
%
%
%------------
% biblio
%------------
%\newpage
\bibliographystyle{plainnat}
\bibliography{interferometry-theory}
%\nocite{*}
\end{document}