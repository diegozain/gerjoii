\documentclass[a4paper,12pt]{article}
%\documentclass[a4paper,10pt]{scrartcl}

\usepackage[utf8x]{inputenc}
\usepackage{amsfonts}
\usepackage{amsmath,esint}
\usepackage{graphicx}
\usepackage{pdfpages}
\usepackage{sansmath}
\usepackage{hyperref}
\usepackage{natbib}
\usepackage{caption}

\usepackage{tikz}


\definecolor{boiseBlue} {RGB}{29,72,159}
\definecolor{rojoAmor} {RGB}{171,13,4}
\definecolor{moradoAmor} {RGB}{93,8,113}
\definecolor{verdeAmor} {RGB}{98,158,31}
\definecolor{negro} {RGB}{10,10,10}
\definecolor{lgreen} {RGB}{180,210,100}
\definecolor{dblue}  {RGB}{20,66,129}
\definecolor{ddblue} {RGB}{11,36,69}
\definecolor{lred}   {RGB}{220,0,0}
\definecolor{nred}   {RGB}{224,0,0}
\definecolor{norange}{RGB}{230,120,20}
\definecolor{nyellow}{RGB}{255,221,0}
\definecolor{ngreen} {RGB}{98,158,31}
\definecolor{dgreen} {RGB}{78,138,21}
\definecolor{nblue}  {RGB}{28,130,185}
\definecolor{jblue}  {RGB}{20,50,100}

\usepackage{listings}
\usepackage{xcolor}
\usepackage{verbatim}
\lstset{language=C++,
		basicstyle=\ttfamily,
	       backgroundcolor=\color{black!5}\ttfamily,
                keywordstyle=\color{nblue}\ttfamily,
                stringstyle=\color{nred}\ttfamily,
                commentstyle=\color{ngreen}\ttfamily,
                morecomment=[l][\color{moradoAmor}]{\#}
}

\newenvironment{rcases}{\left.\begin{aligned}}{\end{aligned}\right\rbrace}

\renewcommand{\familydefault}{\sfdefault}

\newcommand{\specialcell}[2][c]{%
  \begin{tabular}[#1]{@{}c@{}}#2\end{tabular}}
% \specialcell{Foo\\bar}

\title{Iris-Syscal for Dummies\\{\normalsize Diego Domenzain}}
\author{}
\date{}

\pdfinfo{%
  /Title    ()
  /Author   ()
  /Creator  ()
  /Producer ()
  /Subject  ()
  /Keywords ()
}

\begin{document}
\maketitle
%-------------------
% on/off
%-------------------
\section*{Turn {\it Syscal} on}
\begin{enumerate}
\item If using external battery, connect it to $\pm$ sockets in the {\it Tx} panel and push small lever to ``ext".
\item Pull out the break button on the right side.
\item Next to the {\it Rx} panel, push small lever to ``on".
\item Turn knob under on/off lever and leave it a bit open (for ventilation of internal system).
\end{enumerate}
%-------------------
% before
%-------------------
\section*{Before field campaign}
\begin{enumerate}
\item Create survey type as file \texttt{abmn.txt}. This file is a list of {\it all} source-receiver pairs with $xyz$ coordinates for each electrode. The joint-inversion Matlab package {\bf gerjoii} can do this: \texttt{gerjoii2iris\_dc.m}.
\item Turn {\it Syscal} and laptop on.
\item Connect laptop to {\it Syscal} through USB and port ``com 1" in the {\it Rx} panel.
\item In laptop open {\it Electrepro} and follow,
\item \texttt{Open$\to$abmn.txt},
\item \texttt{Upload}. There is a shortcut button for \texttt{Upload} that looks like $\equiv\square$.
\item In {\it Syscal} follow \texttt{Sequence$\to$Upload PC} and write down the memory \#.
\end{enumerate}
%-------------------
% during
%-------------------
\section*{During field campaign}
\begin{enumerate}
\item Plug in electrode cable and turn {\it Syscal} on. 
\\\\
NOTE: {\it Syscal} reads the cable \texttt{1-36} beginning from electrode furthest of {\it Syscal}, so if survey consists of $n_{elect}<36$ then start counting from the end of the cable onwards and leave remaining entries in cable without electrode connection.
\item In the main menu: \texttt{Config$\to$Mode$\to$Change$\to$Automatic sequence$\to$Choose survey}.
\item Put in your sequence (from the \texttt{abmn.txt} file) and choose \texttt{Switch type} to \texttt{internal switch pro}.
\item In the main menu: \texttt{Config$\to$Name} and put your name, i.e. \texttt{ASTERIX}.
\item Check specific parameters in \texttt{Config$\to$},
\begin{itemize}
\item \texttt{Stack} and choose \texttt{Quality factor}.
\item \texttt{Options} and choose source type (ip or er) and \texttt{signed/un-signed} voltage values.
\item \texttt{Tx.param} and choose $\rho$ and time of injection.
\item \texttt{Tx.param$\to$Vab requested} fixed in 800V.
\item \texttt{El. array} (i.e. \texttt{ASTERIX}) and choose,
\begin{align*}
&\texttt{mixed/poly dip}\\
&\texttt{no channel: 10}.
\end{align*}
\item \texttt{Skip elect} and choose,
\begin{align*}
&\texttt{first: 1, last: $n_{elect}$} 
\end{align*}
where $n_{elect}$ is total number of electrodes. 
\end{itemize}
\item DO NOT do: \texttt{Config$\to$Load default}.
\item In menu press \texttt{Tools$\to$RS-CHECK}, then press \texttt{START} (on the keyboard!!) and write down memory \#. This will initiate the survey.
\end{enumerate}
%-------------------
% electrode malfunction
%-------------------
\clearpage
\section*{\texttt{RS-CHECK} gets stuck}
In the case \texttt{RS-CHECK} gives 999.99 kOhms for all electrodes, chances are either your cable and/or cable-connector are broken. To trouble-shoot this issue:
\begin{enumerate}
\item Connect a banana cable to electrode position {\it P2} in the {\it Syscal} top panel.
\item In the {\it Syscal} menu go to \texttt{Check Switch$\to$Check each electrode} and check each electrode position on your cable with the other end of the banana cable.
\end{enumerate}
Keep in mind that {\it Syscal} counts electrodes in reverse from the {\it Syscal}. If you are using one 36 electrode cable {\it Syscal} will look for electrodes 36-19 and NOT 1-18.
%-------------------
% after
%-------------------
\clearpage
\section*{After field campaign}
\begin{enumerate}
\item Turn {\it Syscal} and laptop on and connect them with USB to ``com 1".
\item In laptop open {\it Prosys II} and do: \texttt{Communication$\to$Data Download$\to$Syscal Pro}.
\item In {\it Syscal} open \texttt{Memory$\to$Explore} and find memory interval \#'s where the data was written.
\item Write memory interval in {\it Prosys II}.
\item In {\it Syscal} do: \texttt{Memory$\to$Data Download}.
\item Rename your \texttt{csv} output file to \texttt{your-survey.csv}.
\end{enumerate}
%-------------------
% pre-processing
%-------------------
\section*{Looking at the data}
\begin{enumerate}
\item Turn your laptop on and go to \texttt{ER-gerjoii/raw/your-survey/dc-data/} and put \texttt{your-survey.csv} there.
\item Edit \texttt{your-survey.csv} file to have no spaces in the first column and save it as \texttt{your-survey.txt}.
\item Open two terminals in directory \texttt{ER-gerjoii/dc-processing/}: one with Matlab and one plain.
\item Edit \texttt{dc\_process.py} to match your directory and file names.
\item In the plain terminal do \texttt{python dc\_process.py}.
\item Edit \texttt{datavis\_dc2.m} for the std cutoff and $a$-spacing you want to plot for the dipole-dipole surveys.
\item In the Matlab terminal do \texttt{datavis\_dc2;}.
\item Comment on how the data looks while drinking coffee and discuss cut-off std.
\item Once you are ready to use it in the inversion, in the Matlab terminal do \texttt{iris2gerjoii\_dc;}.
\end{enumerate}
%
%
%
%------------
% biblio
%------------
%\newpage
\bibliographystyle{plainnat}
\bibliography{interferometry-theory}
%\nocite{*}
\end{document}