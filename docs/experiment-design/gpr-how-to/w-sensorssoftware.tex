\documentclass[a4paper,12pt]{article}
%\documentclass[a4paper,10pt]{scrartcl}

\usepackage[utf8x]{inputenc}
\usepackage{amsfonts}
\usepackage{amsmath,esint}
\usepackage{graphicx}
\usepackage{pdfpages}
\usepackage{sansmath}
\usepackage{hyperref}
\usepackage{natbib}
\usepackage{caption}

\usepackage{tikz}


\definecolor{boiseBlue} {RGB}{29,72,159}
\definecolor{rojoAmor} {RGB}{171,13,4}
\definecolor{moradoAmor} {RGB}{93,8,113}
\definecolor{verdeAmor} {RGB}{98,158,31}
\definecolor{negro} {RGB}{10,10,10}
\definecolor{lgreen} {RGB}{180,210,100}
\definecolor{dblue}  {RGB}{20,66,129}
\definecolor{ddblue} {RGB}{11,36,69}
\definecolor{lred}   {RGB}{220,0,0}
\definecolor{nred}   {RGB}{224,0,0}
\definecolor{norange}{RGB}{230,120,20}
\definecolor{nyellow}{RGB}{255,221,0}
\definecolor{ngreen} {RGB}{98,158,31}
\definecolor{dgreen} {RGB}{78,138,21}
\definecolor{nblue}  {RGB}{28,130,185}
\definecolor{jblue}  {RGB}{20,50,100}

\usepackage{listings}
\usepackage{xcolor}
\usepackage{verbatim}
\lstset{language=C++,
		basicstyle=\ttfamily,
	       backgroundcolor=\color{black!5}\ttfamily,
                keywordstyle=\color{nblue}\ttfamily,
                stringstyle=\color{nred}\ttfamily,
                commentstyle=\color{ngreen}\ttfamily,
                morecomment=[l][\color{moradoAmor}]{\#}
}

\newenvironment{rcases}{\left.\begin{aligned}}{\end{aligned}\right\rbrace}

\usepackage{titling}
\setlength{\droptitle}{-5em}   % This is your set screw

\renewcommand{\familydefault}{\sfdefault}

\newcommand{\specialcell}[2][c]{%
  \begin{tabular}[#1]{@{}c@{}}#2\end{tabular}}
% \specialcell{Foo\\bar}

\title{Fiber optic GPR for Dummies\\{\normalsize Diego Domenzain}}
\author{}
\date{}

\pdfinfo{%
  /Title    ()
  /Author   ()
  /Creator  ()
  /Producer ()
  /Subject  ()
  /Keywords ()
}

\begin{document}
\maketitle
%-------------------
% on/off
%-------------------
\section*{Making it work}
\begin{enumerate}
\item Assemble antennas
\begin{itemize}
\item Make sure they turn on by pushing the on/off button and watching a bright red light appear on their led thingy
\item Turn them off
\end{itemize}
\item Connect antennas to DVL
\begin{itemize}
\item Tx (source) is lower entry in DVL:
\begin{itemize}
\item Top plug (light gray) goes to ``input" in antenna (dark gray)
\item Do NOT connect low plug
\end{itemize}
%
\item Rx (receiver) is upper entry in DVL:
\begin{itemize}
\item Top plug (light gray) goes to ``input" in antenna (dark gray)
\item Low plug (dark gray) goes to ``output" in antenna (light gray)
\end{itemize}
\item You SHOULD check which cable is which by putting light in one end and seeing on which end comes out. {\it Some} people label them flipped.
\end{itemize}
\item Turn DVL on
\item Under \texttt{GPR Parameters} choose \texttt{Pulser setting} to \texttt{PE100 1K}
\end{enumerate}
%
\begin{table}[!h]
\centering
\begin{tabular}{ r | l | l}
$f_o$ MHz & $\lambda_o$ m & $\lambda_o/4$ m \\
\hline
50 & 2 & 0.5 \\
100 & 1 & 0.2 \\
200 & 0.5 & 0.1 \\
250 & 0.5 & 0.1 \\
500 & 0.2 & 0.05 \\
\hline
\end{tabular}
\caption{Approximate wavelengths for $\varepsilon_{max}$ = 9 ($v_{min}$ = 0.1 m/ns). You can do your own by computing $\lambda_o=300/f_{o,MHz}/\sqrt{\varepsilon_{max}}$.}
 \label{tbl:wavelength}
 \end{table}
%
%
%
\begin{table}[!h]
\centering
\begin{tabular}{ r | l | l}
$f_o$ MHz & $\lambda_o$ m & $\lambda_o/4$ m \\
\hline
50 & 1 & 0.2 \\
100 & 0.5 & 0.1 \\
200 & 0.25 & 0.05 \\
250 & 0.2 & 0.05 \\
500 & 0.1 & 0.02 \\
\hline
\end{tabular}
\caption{Approximate wavelengths for $\varepsilon_{max}$ = 30 ($v_{min}$ = 0.05 m/ns). You can do your own by computing $\lambda_o=300/f_{o,MHz}/\sqrt{\varepsilon_{max}}$.}
 \label{tbl:wavelength}
 \end{table}
%
%
%------------
% biblio
%------------
%\newpage
\bibliographystyle{plainnat}
\bibliography{interferometry-theory}
%\nocite{*}
\end{document}