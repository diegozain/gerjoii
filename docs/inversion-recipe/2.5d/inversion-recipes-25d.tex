\documentclass[a4paper,12pt]{article}
%\documentclass[a4paper,10pt]{scrartcl}

\usepackage[utf8x]{inputenc}
\usepackage{amsfonts}
\usepackage{amsmath,esint}
\usepackage{graphicx}
\usepackage{pdfpages}
\usepackage{sansmath}
\usepackage{hyperref}
\usepackage{natbib}
\usepackage{caption}

\usepackage{tikz}


\definecolor{boiseBlue} {RGB}{29,72,159}
\definecolor{rojoAmor} {RGB}{171,13,4}
\definecolor{moradoAmor} {RGB}{93,8,113}
\definecolor{verdeAmor} {RGB}{98,158,31}
\definecolor{negro} {RGB}{10,10,10}
\definecolor{lgreen} {RGB}{180,210,100}
\definecolor{dblue}  {RGB}{20,66,129}
\definecolor{ddblue} {RGB}{11,36,69}
\definecolor{lred}   {RGB}{220,0,0}
\definecolor{nred}   {RGB}{224,0,0}
\definecolor{norange}{RGB}{230,120,20}
\definecolor{nyellow}{RGB}{255,221,0}
\definecolor{ngreen} {RGB}{98,158,31}
\definecolor{dgreen} {RGB}{78,138,21}
\definecolor{nblue}  {RGB}{28,130,185}
\definecolor{jblue}  {RGB}{20,50,100}

\usepackage{listings}
\usepackage{xcolor}
\usepackage{verbatim}
\lstset{language=C++,
		basicstyle=\ttfamily,
	       backgroundcolor=\color{black!5}\ttfamily,
                keywordstyle=\color{nblue}\ttfamily,
                stringstyle=\color{nred}\ttfamily,
                commentstyle=\color{ngreen}\ttfamily,
                morecomment=[l][\color{moradoAmor}]{\#}
}
\newenvironment{rcases}{\left.\begin{aligned}}{\end{aligned}\right\rbrace}
\renewcommand{\familydefault}{\sfdefault}
\newcommand{\specialcell}[2][c]{%
  \begin{tabular}[#1]{@{}c@{}}#2\end{tabular}}
% \specialcell{Foo\\bar}

\usepackage{titling}
\setlength{\droptitle}{-10em}

\title{2.5d inversion recipes}
\author{}
\date{}

\pdfinfo{%
  /Title    ()
  /Author   ()
  /Creator  ()
  /Producer ()
  /Subject  ()
  /Keywords ()
}

\begin{document}
\maketitle
%-------------------
% main flow
%-------------------
%
%\begin{figure}[!h]
%\centering
%% left low right up
%\includegraphics[trim={50 300 50 280},clip,width=1\textwidth]{pics/sig.pdf}
%\caption{{\bf Top:} My inversion. {\bf Bottom:} Pygimli inversion.}
%\end{figure}
%
% --------
% 
% --------
\section*{ER inversion}
This process is repeated for each experiment (i.e. each source $s$ and all its receivers). Let the 2.5d weights for source $s$ be $\{k,\omega\}$ and stored in disk,
\begin{enumerate}
\item compute synthetic electric potentials $\{\tilde{u}_i\}$, data $d$ and error $e=d-d^o$,
\begin{enumerate}
\item retrieve from memory $k$ and $\omega$,
\item choose $k_i\in k$ and build $L_i$ with the right boundary conditions for that $k_i$,
\begin{align*}
L^{i} &\approx -\nabla\cdot\sigma\nabla,\\
L_{i} &= L^{i} + k_i^2\,\sigma,
\end{align*}
\item solve $L_{i}\tilde{u}_{i}=\frac{s}{2}$ for $\tilde{u}_{i}$ and store,
\item after all $k$ has been used,
\[ 
u=\sum_i \omega_i \,\tilde{u}_{i} \to d=Mu \to e=d-d^o.
\]
\end{enumerate}
\item correct error amplitude by its variance,
\[
e\gets \frac{e}{1+\operatorname{std}\,(d^o)}
\]
\item compute $g$,
\begin{enumerate}
\item choose $k_i\in k$,
\item build $L_i$ and $S_i=-((\nabla_\sigma L^i)\tilde{u}_i)^\top - k_i^2 \, \mbox{diag}(\tilde{u}_i)^\top$
\item solve $L_i^\top a_i=M^\top e$ for $a_i$
\item compute and store $g_i=S_i a_i$,
\item after all $k$ has been used,
\[ 
g=\frac{2}{\pi}\sum \omega_i g_i.
\]
\end{enumerate}
\item On $g$ filter out high spatial-frequencies and normalize by largest amplitude,
\item compute $\sigma_{err}=\sigma-\sigma_{ref}$, filter out high spatial-frequencies and normalize by largest amplitude,
\item $g\gets g+\beta\sigma_{err}$ and normalize $g$ by largest amplitude,
\item compute step-size $\alpha_s$ using Pica and set $\Delta\sigma_s=-\alpha_s g$
\item store $\Delta\sigma_s$ and $\alpha_s$.
\end{enumerate}
After all sources have their update direction,
\begin{align*}
\alpha &= \frac{1}{n_s} \sum\alpha_s \\
\Delta\sigma &= \sum\Delta\sigma_s \\
\Delta\sigma_{dc} &= \alpha\frac{\Delta\sigma}{\max | \Delta\sigma |}.
\end{align*}
The conductivity is updated as,
\begin{align*}
\sigma\gets\sigma\odot\exp\{\sigma\odot\Delta\sigma_{dc}\}.
\end{align*}
% --------
% 
% --------
\section*{GPR inversion $\varepsilon$}
This process is repeated for each experiment (i.e. each source $s$ and all its receivers),
\begin{enumerate}
\item compute synthetic data $d$ and error $e=d-d^o$,
\item \color{red} how to treat noise in $e$?\color{black}
\item compute $g_{\varepsilon}$ and apply Kurzmann preconditioner,
\item on $g_{w,\varepsilon}$ filter out high spatial-frequencies and normalize by largest amplitude,
\item compute step-size $\alpha$ using Pica and set $\Delta\varepsilon_s=-\alpha g_\varepsilon$
\item store $\Delta\varepsilon_s$,
\item invert for source wavelet with a \color{red} Wiener filter\color{black}.
\end{enumerate}
After all sources have their update direction, 
\begin{align*}
\Delta\varepsilon &= \frac{1}{n_s} \sum\Delta\varepsilon_s.
\end{align*}
The permittivity is updated as,
\begin{align*}
\varepsilon\gets\varepsilon\odot\exp\{\varepsilon\odot\Delta\varepsilon\}.
\end{align*}
% --------
% 
% --------
\section*{GPR inversion $\sigma$}
This process is repeated for each experiment (i.e. each source $s$ and all its receivers),
\begin{enumerate}
\item compute synthetic data $d$ and error $e=d-d^o$,
\item \color{red} how to treat noise in $e$?\color{black}
\item compute $g_{w,\sigma}$ and apply Kurzmann preconditioner,
\item on $g_{w,\sigma}$ filter out high spatial-frequencies and normalize by largest amplitude,
\item compute step-size $\alpha$ using Pica and set $\Delta\sigma_s=-\alpha g_{w,\sigma}$
\item store $\Delta\sigma_s$,
\item invert for source wavelet with \color{red} Wiener filter\color{black}.
\end{enumerate}
After all sources have their update direction, 
\begin{align*}
\Delta\sigma_w &= \frac{1}{n_s} \sum\Delta\sigma_s.
\end{align*}
The conductivity is updated as,
\begin{align*}
\sigma\gets\sigma\odot\exp\{\sigma\odot\Delta\sigma_w\}.
\end{align*} 
% --------
% 
% --------
\section*{GPR+ER inversion $\sigma$}
The update direction is 
\begin{align*}
\Delta\sigma\gets \frac{1}{2} (\Delta\sigma_w + \Delta\sigma_{dc}) 
\end{align*} 
and the update for {\it both} the GPR and ER is
\begin{align*}
\sigma\gets\sigma\odot\exp\{\sigma\odot\Delta\sigma\}.
\end{align*} 
%
%------------
% biblio
%------------
%\newpage
% \bibliographystyle{plainnat}
% \bibliography{w-2d-25d-src}
%\nocite{*}
\end{document}