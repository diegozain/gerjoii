\documentclass{beamer}
\usepackage{tikz}
\usetikzlibrary{arrows,positioning,decorations.pathreplacing} 

\usepackage[utf8x]{inputenc}
\usepackage{amsfonts}
\usepackage{amsmath,esint}
\usepackage{graphicx}
\usepackage{multimedia}
\usepackage{amssymb}
%
\usepackage{animate}
 \graphicspath{{"/pngs"}}
%
\usepackage{epstopdf}
\usepackage{pdfpages}
\usepackage{dsfont}
\usepackage{textpos}
\usepackage[font=it]{caption}
\usepackage{bbm}
\usepackage{varwidth}
\usepackage{media9}
\usepackage{multimedia}

%\usetheme{Madrid}
\beamertemplatenavigationsymbolsempty
\setbeamertemplate{footline}[page number]{}
% \usecolortheme{wolverine}
\useinnertheme{rounded}

\definecolor{boiseBlue} {RGB}{29,72,159}
\definecolor{rojoAmor} {RGB}{171,13,4}
\definecolor{moradoAmor} {RGB}{93,8,113}
\definecolor{verdeAmor} {RGB}{98,158,31}
\definecolor{negro} {RGB}{10,10,10}
\definecolor{lgreen} {RGB}{180,210,100}
\definecolor{dblue}  {RGB}{20,66,129}
\definecolor{ddblue} {RGB}{11,36,69}
\definecolor{lred}   {RGB}{220,0,0}
\definecolor{nred}   {RGB}{224,0,0}
\definecolor{norange}{RGB}{230,120,20}
\definecolor{nyellow}{RGB}{255,221,0}
\definecolor{ngreen} {RGB}{98,158,31}
\definecolor{dgreen} {RGB}{78,138,21}
\definecolor{nblue}  {RGB}{28,130,185}
\definecolor{jblue}  {RGB}{20,50,100}

\setbeamercolor{palette primary}{fg=white,bg=black}
\setbeamercolor{frametitle}{fg=moradoAmor,bg=white}
\setbeamercolor{title page}{fg=nblue,bg=black}
\setbeamertemplate{itemize items}[circle]

\defbeamertemplate*{title page}{customized}[1][]
{
\begin{beamercolorbox}[sep=1em,wd=\textwidth,ht=8em,dp=-1em,center]{boldfont}
\usebeamerfont{title} {\center \color{norange} \Large \textbf \inserttitle}
\end{beamercolorbox}
\usebeamercolor{author}{\center \color{black}\normalsize{\insertauthor}\\[2em]}
\usebeamercolor{institute in headline}{\center \color{black}\normalsize{\insertinstitute}\\[5em]}
\usebeamercolor{date}{\center \color{black}\small{\insertdate}\\[0em]}
}


\makeatletter
    \newenvironment{withoutheadline}{
        \setbeamertemplate{headline}[default]
        \def\beamer@entrycode{\vspace*{-\headheight}}
    }{}
\makeatother

% Make lists without bullets
\newenvironment{itemizemine}{
  \begin{list}{}{
    \setlength{\leftmargin}{5em}
    \color{nred}
    \itemindent=3em
  }
}{
  \end{list}
}


\title{
Imaging wavefields using interferometry
}
\author{{\bf Diego Domenzain}}
\institute{
\begin{tikzpicture}[remember picture, overlay] 
\node [] at (0,0) 
{\includegraphics[height=0.7cm]{bsuwiki.png}}; 
\end{tikzpicture}
}
\date{\vspace{1em}Spring 2017}

\begin{document}

\frame{\titlepage}
%--------------------------------------------------------------------
%            begining                                                                                       
%--------------------------------------------------------------------
\frame[b]
{
\centering
\includegraphics[width=1\textwidth]{tikz-pics/outline-real.pdf}
\centering
\vfill
\begin{varwidth}{0.8\textwidth}
\centering
\color{black}{\bf How can we image time traces of wavefields from passive source 
observations?}
\end{varwidth}
%\vfill
%\color{nred} Although this talk will be about {\it how} it works, not {\it how well}.

}
%--------------------------------------------------------------------
%            motivation                                                                                  
%--------------------------------------------------------------------
\frame
{
\centering
\color{norange}{\bf motivation}
}
%--------------------------------------------------------------------
%            motivation		goals and caveats                                                                                     
%--------------------------------------------------------------------
\frame
{
\frametitle{{\bf Goal and challenges}}
We ultimately want to image material properties using wave energy.
\\~\\
Sometimes actively generating a wavefield is not an option.
\\~\\
Some others, there are already many wavefields propagating through our zone of interest.
\\~\\
\color{nred}How do we use these wavefields?\color{black}
}
%--------------------------------------------------------------------
%            motivation		listen                                                                                     
%--------------------------------------------------------------------
\frame
{
\frametitle{{\bf Listen}}
The idea is to,
\\~\\
\begin{itemize}
\item listen to echoes of wavefields generated by many sources,
\item correct these records of echoes,
\item image material properties using the corrected echoes.
\end{itemize} 
\vfill
\flushright
... so, \color{nred}reorganize many energy responses into a coherent response.\color{black}
}
%--------------------------------------------------------------------
%            motivation		example from paper                                                                   
%--------------------------------------------------------------------
\frame
{
\frametitle{{\bf Imaging a wavefield in real life}}
\centering
\includegraphics[width=0.8\textwidth]{tikz-pics/us-array.png}
\vfill
\flushright{
Lin et al.
}
}
%--------------------------------------------------------------------
%            method                                                                                 
%--------------------------------------------------------------------
\frame
{
\centering
\color{norange}{\bf method}
}
%--------------------------------------------------------------------
%            method		best case                                                                                     
%--------------------------------------------------------------------
\frame
{
\frametitle{{\bf Best case scenario}}
\includegraphics[width=1\textwidth]{tikz-pics/svg/interferometry-0.pdf}
}
%--------------------------------------------------------------------
%            method		code                                                                                    
%--------------------------------------------------------------------
\frame
{
\frametitle{{\bf Pseudo-code}}
\includegraphics[width=1\textwidth]{tikz-pics/svg/interferometry-1.pdf}
}
%--------------------------------------------------------------------
%            method		example      setup                                                                               
%--------------------------------------------------------------------
\frame
{
\frametitle{{\bf Example - setup}}
\begin{columns}
\begin{column}{0.5\textwidth}
\includegraphics[trim = 40mm 80mm 40mm 80mm, clip,width=1\textwidth]{tikz-pics/2/1000sources.pdf}
\end{column}
\begin{column}{0.5\textwidth}
\includegraphics[trim = 40mm 80mm 40mm 80mm, clip,width=1\textwidth]{tikz-pics/2/wanted.pdf}
\end{column}
\end{columns}
}
%--------------------------------------------------------------------
%            method		example             stacking                                                                        
%--------------------------------------------------------------------
\frame
{
\frametitle{{\bf Example - stacking}}
\movie[width=1\textwidth,height=0.5\textwidth,poster]{}{interferate.avi}
}
%--------------------------------------------------------------------
%            method		example             end result                                                                        
%--------------------------------------------------------------------
\frame
{
\frametitle{{\bf Example - virtual shot gather}}
\begin{columns}
\begin{column}{0.5\textwidth}
\includegraphics[trim = 40mm 80mm 40mm 80mm, clip,width=1\textwidth]{tikz-pics/2/full-virt-trace.pdf}
\end{column}
\begin{column}{0.5\textwidth}
\includegraphics[trim = 40mm 80mm 40mm 80mm, clip,width=1\textwidth]{tikz-pics/2/short-virt-trace.pdf}
\end{column}
\end{columns}
}
%--------------------------------------------------------------------
%            theory                                                                                 
%--------------------------------------------------------------------
\frame
{
\centering
\color{norange}{\bf theory}
}
%--------------------------------------------------------------------
%            theory		behind the stage                                                                                    
%--------------------------------------------------------------------
\frame
{
\frametitle{{\bf Behind the stage}}
\includegraphics[width=1\textwidth]{tikz-pics/svg/interferometry-2.pdf}
}
%--------------------------------------------------------------------
%            theory		sources are receivers                                                                                    
%--------------------------------------------------------------------
\frame
{
\frametitle{{\bf Source-receiver duality}}
\includegraphics[width=1\textwidth]{tikz-pics/svg/interferometry-3.pdf}
}
%--------------------------------------------------------------------
%            theory		fwi                                                                                   
%--------------------------------------------------------------------
\frame
{
\frametitle{{\bf Full {\it traceform} inversion}}
Recall the FWI scheme,
\begin{align*}
\int_{0}^{T} u(T-t)\, q(t) \;{\rm d}t &= \nabla_{\sigma} {\sf E},
\end{align*}
where \color{nblue}$q$~\color{black} is the forward ~\color{nblue}back-propagation of errors~\color{black} and 
\color{nblue}$u$~\color{black} is our \color{nblue}generated wavefield\color{black}.
\\~\\
Now, think of \color{nblue}$s$~\color{black} as the \color{nblue}spatial boundary\color{black}, 
\color{nblue}$r_{\bullet}$~\color{black} as our \color{nblue}source location\color{black}, 
and the rest of receivers \color{nblue}$r_i$~\color{black} as our \color{nblue}receivers~\color{black} in our FWI scheme.
\\~\\
We have,
\begin{align*}\int_{\Gamma} \underbrace{g(r_{\bullet},s)}_{u(T-t)} * \underbrace{g(s,r_i)}_{q(t)} \;{\rm d}\Gamma \;\; &\propto \;\;
g(r_i,r_{\bullet},-t) + g(r_i,r_{\bullet},t).
\end{align*}
}
%--------------------------------------------------------------------
%            theory		actual implementation                                                                                    
%--------------------------------------------------------------------
\frame
{
\frametitle{{\bf Pseudo-code of implementation}}
\includegraphics[width=1\textwidth]{tikz-pics/svg/interferometry-1-2.pdf}
}
%--------------------------------------------------------------------
%            sensitivity                                                                                
%--------------------------------------------------------------------
\frame
{
\centering
\color{norange}{\bf sensitivity}
}
%--------------------------------------------------------------------
%           sensitivity		Correlation gather                                                                              
%--------------------------------------------------------------------
\frame
{
\frametitle{{\bf Correlation gather}}
\begin{columns}
\begin{column}{0.5\textwidth}
\includegraphics[trim = 40mm 80mm 40mm 80mm, clip,width=1\textwidth]{tikz-pics/2/1000sources.pdf}
\end{column}
\begin{column}{0.5\textwidth}
\includegraphics[trim = 40mm 80mm 40mm 80mm, clip,width=1\textwidth]{tikz-pics/2/corr-gath.pdf}
\end{column}
\end{columns}
}
%--------------------------------------------------------------------
%            sensitivity		Fresnel zones and curvature of s                                                                        
%--------------------------------------------------------------------
\frame
{
\frametitle{{\bf Fresnel zones and curvature of $s$}}
\begin{columns}
\begin{column}{0.4\textwidth}
\includegraphics[width=1\textwidth]{tikz-pics/fresnel-zones.png}
\end{column}
\begin{column}{0.4\textwidth}
\includegraphics[width=1\textwidth]{tikz-pics/corr-gath-article.png}
\end{column}
\end{columns}
\vspace{2em}
\begin{itemize}
\item Sources parallel to the line segment $X_A - X_B$ give extrema in the correlation gather.
\item The closer $X_A$ and $X_B$ are to $s$, the more they feel its curvature, the narrower the Fresnel zone.
\end{itemize}
\vfill
\flushright{
Wapenaar et al.
}
}
%--------------------------------------------------------------------
%            sensitivity		Fresnel zones and w                                                                       
%--------------------------------------------------------------------
\frame
{
\frametitle{{\bf Fresnel zones and $\omega$}}
\begin{columns}
\begin{column}{0.5\textwidth}
\centering
\small{$600\;[MHz]$}
\includegraphics[trim = 40mm 80mm 40mm 80mm, clip,width=1\textwidth]{tikz-pics/600Mhz/corr-gath-tight.pdf}
\end{column}
\begin{column}{0.5\textwidth}
\centering
\small{$1000\;[MHz]$}
\includegraphics[trim = 40mm 80mm 40mm 80mm, clip,width=1\textwidth]{tikz-pics/1000Mhz/corr-gath-tight.pdf}
\end{column}
\end{columns}
}
%--------------------------------------------------------------------
%           sensitivity		what does it mean?                                                                        
%--------------------------------------------------------------------
\frame
{
\frametitle{{\bf Known issues}}
\begin{itemize}
\item Heterogeneous media gives rise to spurious events in virtual shot gathers.
\begin{columns}
\begin{column}{0.4\textwidth}
\includegraphics[width=1\textwidth]{tikz-pics/circle-dylan.png}
\end{column}
\begin{column}{0.4\textwidth}
\includegraphics[width=1\textwidth]{tikz-pics/virtual-refraction.png}
\end{column}
\end{columns}
\item Optimal $(s,r)$ geometry is not always ensured.
\\~\\
\item How to crop original $(r,t)$ gather into $(r,s,t)$ cube?
\end{itemize}
}
%--------------------------------------------------------------------
%           the end                                                                      
%--------------------------------------------------------------------
\frame
{
\centering
\Huge{?}
}
\end{document}

%
%
%----
% frame
%----
\withoutheadline
{
}
\frame
{
\frametitle{stuff}
}
%
%
\begin{columns}
\begin{column}{0.5\textwidth}

\end{column}
\begin{column}{0.5\textwidth} 

\end{column}
\end{columns}
%
%
\includegraphics[height=3cm,trim={0 0 460 0},clip]{soap_film2.jpg} % trim=left bottom right top
%
%
\iffalse
\fi


