\documentclass[a4paper,12pt]{article}
%\documentclass[a4paper,10pt]{scrartcl}

\usepackage[utf8x]{inputenc}
\usepackage{amsfonts}
\usepackage{amsmath,esint}
\usepackage{graphicx}
\usepackage{pdfpages}
\usepackage{sansmath}
\usepackage{hyperref}
\usepackage{natbib}
\usepackage{bbm}
\usepackage{xcolor}

\definecolor{boiseBlue} {RGB}{29,72,159}
\definecolor{rojoAmor} {RGB}{171,13,4}
\definecolor{moradoAmor} {RGB}{93,8,113}
\definecolor{verdeAmor} {RGB}{98,158,31}
\definecolor{negro} {RGB}{10,10,10}
\definecolor{lgreen} {RGB}{180,210,100}
\definecolor{dblue}  {RGB}{20,66,129}
\definecolor{ddblue} {RGB}{11,36,69}
\definecolor{lred}   {RGB}{220,0,0}
\definecolor{nred}   {RGB}{224,0,0}
\definecolor{norange}{RGB}{230,120,20}
\definecolor{nyellow}{RGB}{255,221,0}
\definecolor{ngreen} {RGB}{98,158,31}
\definecolor{dgreen} {RGB}{78,138,21}
\definecolor{nblue}  {RGB}{28,130,185}
\definecolor{jblue}  {RGB}{20,50,100}

\usepackage{listings}
\usepackage{xcolor}
\lstset{language=C++,
		basicstyle=\ttfamily,
	       backgroundcolor=\color{black!5}\ttfamily,
                keywordstyle=\color{nblue}\ttfamily,
                stringstyle=\color{nred}\ttfamily,
                commentstyle=\color{ngreen}\ttfamily,
                morecomment=[l][\color{moradoAmor}]{\#}
}

\newenvironment{rcases}{\left.\begin{aligned}}{\end{aligned}\right\rbrace}
\renewcommand{\labelitemii}{$\circ$}
\renewcommand{\familydefault}{\sfdefault}

\newcommand{\specialcell}[2][c]{%
  \begin{tabular}[#1]{@{}c@{}}#2\end{tabular}}
% \specialcell{Foo\\bar}

\title{\Huge{
\color{nblue}g\color{nred}er\color{ngreen}j\color{nblue}o\color{nred}i\color{moradoAmor}i}
\color{black}\\
Imaging using EM waves \\ \small{summer 2017}}
\author{}
\date{}

\pdfinfo{%
  /title    ()
  /Author   ()
  /Creator  ()
  /Producer ()
  /Subject  ()
  /Keywords ()
}

\begin{document}
\maketitle
%-------------------
% fwd
%-------------------
\section{Forward}
Using the finite difference scheme solve for $u$,
\begin{align*}
\mathbf{v} &= (H_z,\; -H_x)^\top & u=E_y \\
 {\bf s}&=(0,\;0,\;-J_y)^\top & {\bf w=(v,}\;u)^\top
\end{align*}
%
%
%
\begin{align*}
%
\underbrace{
\begin{bmatrix}
\mu_o \mathbbm{1}_2 & 0 \\
0 & \varepsilon
\end{bmatrix}
}_{A}
%
{\bf \dot{w}}
&=
%
\underbrace{
\begin{bmatrix}
0 & \nabla^\top \\
\nabla & 0
\end{bmatrix}
}_{\mathrm{D}}
%
{\bf w} -
%
\underbrace{
\begin{bmatrix}
0 & 0 \\
0 & \sigma
\end{bmatrix}
}_B
%
{\bf w} +
%
{\bf s},\\
\mathcal{L} &= A-{\rm D}+B.
\end{align*}
over a rectangular region $\Omega$ (simulating a slice in depth of the earth) with absorbing boundary 
conditions everywhere. Matrix $A$ controls {\it velocity}, matrix $B$ controls {\it attenuation}.
\\\\
Below is a brief list/explanation of what each method does. Some methods do more than explained here, 
take more parameters, and output more stuff. This is just for a comprehensive idea.
\begin{itemize}
\item Build geometric parameters from target material properties and chosen characteristic frequency,
\begin{align*}
\varepsilon,\, \sigma, \,
\left\{
\begin{matrix}
\Delta x,\, \Delta z, \, \Delta t \\
n,\, m, \, nt \\
x, \, z, \, t
\end{matrix}
\right\}
&= 
\texttt{w\_geom}(c_o,\, \varepsilon,\,\sigma,\, f_{Ny},\, xz_{\text{plane}})
\end{align*}
%
\item Build $M$. In this case, $M$ are just coordinates of receivers,
\[
Mu = u_r.
\]
%
\item Compute wave and error,
\begin{align*}
u, \, e,\,
\left\{
\begin{matrix}
c{\bf J},\,c{\bf E},\,c{\bf H} \\
\text{cpml}\,{\bf E},\, \text{cpml}\,{\bf H}
\end{matrix}
\right\}
&= 
\texttt{w\_fwd}\left(
\varepsilon,\,\sigma,\, {\bf s},\, M,\, d^o,\,
\left\{
\begin{matrix}
\Delta x,\, \Delta z, \, \Delta t \\
n,\, m, \, nt \\
x, \, z, \, t
\end{matrix}
\right\}
\right).
\end{align*}
this method needs some lengthy constructs before it actually solves the wave. the order is as below.
\begin{itemize}
%
\item Expand $\varepsilon,\sigma$ to be in PML,
\begin{align*}
\varepsilon,\,\sigma
&= 
\texttt{w\_exp2pml}\left(
\varepsilon,\,\sigma,\,
\left\{
\begin{matrix}
\Delta x,\, \Delta z, \, \Delta t \\
n,\, m, \, nt \\
x, \, z, \, t
\end{matrix}
\right\}
\right).
\end{align*}
%
\item Build PML and ${\bf H,E}$ finite difference coefficients in PML,
\begin{align*}
\left\{
\begin{matrix}
\text{cpml}\,{\bf E} \\
\text{cpml}\,{\bf H}
\end{matrix}
\right\}
&= 
\texttt{w\_cpml}\left(
\varepsilon,\,
\left\{
\begin{matrix}
\Delta x,\, \Delta z, \, \Delta t \\
n,\, m, \, nt \\
x, \, z, \, t
\end{matrix}
\right\}
\right).
\end{align*}
%
\item Build coefficients for fields in finite difference scheme,
\begin{align*}
c{\bf J},\,c{\bf E},\,c{\bf H}
&= 
\texttt{w\_coeff}\left(
\varepsilon,\,\sigma,\,
\left\{
\begin{matrix}
\Delta x,\, \Delta z, \, \Delta t \\
n,\, m, \, nt \\
x, \, z, \, t
\end{matrix}
\right\}
\right).
\end{align*}
%
\item Solve for wave
\begin{align*}
u
&= 
\texttt{w\_solve}\left(
M,\, {\bf s}, \,
\left\{
\begin{matrix}
\Delta x,\, \Delta z, \, \Delta t \\
n,\, m, \, nt \\
x, \, z, \, t
\end{matrix}
\right\} ,\,
\left\{
\begin{matrix}
c{\bf J},\,c{\bf E},\,c{\bf H} \\
\text{cpml}\,{\bf E},\, \text{cpml}\,{\bf H}
\end{matrix}
\right\}
\right).
\end{align*}
this last method propagates the wave. At each time-step $t$, 
\begin{align*}
d_t &= Mu_t \hspace{2em} \to \hspace{2em} d = \{d_t \, \text{ for all }t \}.
\end{align*}
\item Compute $e$,
\begin{align*}
e &= d-d^o.
\end{align*}
\end{itemize}
\end{itemize}
%-------------------
% inv
%-------------------
\section{Inverse}
Our continuous objective function is
\begin{align*}
{\sf E}(\varepsilon,\,\sigma) = \int_{0}^\top
(\underbrace{
d(\varepsilon,\,\sigma,\,t)-d^{o}(t)
}_{e})^2
\; {\rm d}t
\end{align*}
We optimize ${\sf E}$ by setting $\frac{{\rm d}}{{\rm d}\varepsilon}{\sf E}$ and 
$\frac{{\rm d}}{{\rm d}\sigma}{\sf E}$ equal to zero and solving for the parameters,
\begin{align*}
&\frac{{\rm d}}{{\rm d}\varepsilon}{\sf E} = \int_{0}^{T} \lambda \dot{u} \; {\rm d}t 
&\frac{{\rm d}}{{\rm d}\sigma}{\sf E} = \int_{0}^{T} \lambda u \; {\rm d}t
\end{align*}
where $\lambda$ is the {\it adjoint} wavefield and $e^*$ is the {\it time reversed error},
\[
\mathcal{L}^* \lambda = e^*, \hspace{2em} \lambda(T)=\dot{\lambda}(T)=0.
\]
%
Below is a brief list/explanation of what each method does. Some methods do more than explained here, 
take more parameters, and output more stuff. This is just for a comprehensive idea.
%
\begin{itemize}
\item Compute $\nabla_{\sigma}{\sf E}^\top$ using cross-correlation of $u$ and its adjoint field $\lambda$,
\begin{align*}
g_{\sigma}
&= 
\texttt{w\_grad}\left(
u,\, e,\,
M,\,
\left\{
\begin{matrix}
\Delta x,\, \Delta z, \, \Delta t \\
n,\, m, \, nt \\
x, \, z, \, t
\end{matrix}
\right\}, \,
\left\{
\begin{matrix}
c{\bf J},\,c{\bf E},\,c{\bf H} \\
\text{cpml}\,{\bf E},\, \text{cpml}\,{\bf H}
\end{matrix}
\right\}
\right).
\end{align*}
%
\begin{itemize}
\item Solve for $\lambda$. Source is error $e$ but time-reversed (${\bf s}=e^*$)
\begin{align*}
\lambda
&= 
\texttt{w\_solve}\left(
\varepsilon,\,\sigma,\, e^*, \,
\left\{
\begin{matrix}
\Delta x,\, \Delta z, \, \Delta t \\
n,\, m, \, nt \\
x, \, z, \, t
\end{matrix}
\right\} ,\,
\left\{
\begin{matrix}
c{\bf J},\,c{\bf E},\,c{\bf H} \\
\text{cpml}\,{\bf E},\, \text{cpml}\,{\bf H}
\end{matrix}
\right\}
\right).
\end{align*}
At each time-step $t$, 
\begin{align*}
g_{t} &= u^{*}_{t} \odot \lambda_{t} &
g_{\sigma} \gets g_{t} + g_{t-1}.
\end{align*}
\end{itemize}
%
\item Compute derivative of $u$ with resect to time,
\begin{align*}
\dot{u} &= \texttt{w\_dt}\left(u,
\left\{
\begin{matrix}
\Delta x,\, \Delta z, \, \Delta t \\
n,\, m, \, nt \\
x, \, z, \, t
\end{matrix}
\right\}
\right).
\end{align*}
\item Compute $\nabla_{\varepsilon}{\sf E}^\top$ using cross-correlation of $\dot{u}$ and its adjoint field 
$\lambda$,
\begin{align*}
g_{\varepsilon}
&= 
\texttt{w\_grad}\left(
\dot{u},\, e,\,
\varepsilon,\,\sigma,\,
\left\{
\begin{matrix}
\Delta x,\, \Delta z, \, \Delta t \\
n,\, m, \, nt \\
x, \, z, \, t
\end{matrix}
\right\}, \,
\left\{
\begin{matrix}
c{\bf J},\,c{\bf E},\,c{\bf H} \\
\text{cpml}\,{\bf E},\, \text{cpml}\,{\bf H}
\end{matrix}
\right\}
\right).
\end{align*}
\item Compute step size $\alpha$ for update. Requires one extra forward run for each.
\begin{align*}
\alpha_\sigma &= \texttt{w\_pica\_s.m}
\left(
\varepsilon,\,\sigma,\, {\bf s},\, M,\, d^o,\,
\left\{
\begin{matrix}
\Delta x,\, \Delta z, \, \Delta t \\
n,\, m, \, nt \\
x, \, z, \, t
\end{matrix}
\right\},\,
e,\, g_\sigma,\, k_\sigma
\right), \\
%
\alpha_\varepsilon &= \texttt{w\_pica\_e.m}
\left(
\varepsilon,\,\sigma,\, {\bf s},\, M,\, d^o,\,
\left\{
\begin{matrix}
\Delta x,\, \Delta z, \, \Delta t \\
n,\, m, \, nt \\
x, \, z, \, t
\end{matrix}
\right\},\,
e,\, g_\varepsilon,\, k_\varepsilon
\right).
\end{align*}
%
For example, 
\begin{align*}
d_\bullet & = d(\varepsilon,\, \sigma + k_\sigma g_{\sigma})-d(\varepsilon,\,\sigma), \\
\alpha_\sigma &= 
k_\sigma
\frac{d_\bullet(:)^\top \cdot e_w(:)}{d_\bullet(:)^\top \cdot d_\bullet(:)},
\end{align*}
where $a(:)$ is borrowed from the Matlab command that turns matrix $a$ to vector form.
\end{itemize}
%-------------------
% file names
%-------------------
\newpage
\section{File names}
{\bf Constructors}
\begin{itemize}
\item \texttt{w\_geom.m} builds geometric parameters.
\item \texttt{w\_epsilon.m} build $\varepsilon$.
\item \texttt{w\_sigma.m} build $\sigma$.
\item \texttt{w\_magmat.m} build magnetic parameters (constant and homogeneous) 
$\sigma_{\text{mag}},\mu$.
\item \texttt{w\_exp2pml.m} expand $\varepsilon,\sigma$ inside PML.
\item \texttt{w\_cpml.m} build PML and coefficients for ${\bf E,\, H}$.
\item \texttt{w\_coeff.m} build finite difference coefficients for ${\bf E,\, H,\, J}$.
\item \texttt{w\_s.m} specify source position in index form.
\item \texttt{w\_M.m} specify receiver positions in index form.
\end{itemize}
{\bf Procedures}
\begin{itemize}
\item \texttt{w\_wavelet.m} compute source wavelet.
\item \texttt{w\_solve.m} propagates the wave $u$. This method has local functions,
\begin{itemize}
\item \texttt{new\_Ez.m} + \texttt{new\_Ez\_pml.m} updates $u$,
\item \texttt{new\_H.m} + \texttt{new\_H\_pml.m} updates ${\bf v}$.
\end{itemize}
\item \texttt{w\_fwd.m} computes $u,e$.
\item \texttt{w\_dt.m} computes $\dot{u}$.
\item \texttt{w\_grad.m} computes adjoint $\lambda$ and cross-correlates $u$ with $\lambda$. 
This method does not reference \texttt{w\_solve.m}, it redoes its work, so it also has local functions,
\begin{itemize}
\item \texttt{new\_Ez.m} + \texttt{new\_Ez\_pml.m} updates $u$,
\item \texttt{new\_H.m} + \texttt{new\_H\_pml.m} updates ${\bf v}$.
\end{itemize}
\item \texttt{w\_gd\_s.m} performs gradient descent on ${\sf E}$ solving for $\sigma$.
\item \texttt{w\_gd\_e.m} performs gradient descent on ${\sf E}$ solving for $\varepsilon$.
\item \texttt{w\_pica\_s.m} computes descent stepsize for $g_\sigma$.
\item \texttt{w\_pica\_e.m} computes descent stepsize for $g_\varepsilon$.
\item \texttt{w\_gd\_s\_stoch.m} performs stochastic gradient descent on ${\sf E} = \sum_i{\sf E}_i$ solving for $\sigma$.
\item \texttt{w\_gd\_e\_stoch.m} performs stochastic gradient descent on ${\sf E} = \sum_i{\sf E}_i$ solving for $\varepsilon$.
\end{itemize}
{\bf Examples}
\begin{itemize}
 \item \texttt{w\_run.m}
\item \texttt{w\_run\_many.m}
\item \texttt{w\_inv.m}
\item \texttt{w\_inv\_many.m}
\end{itemize}

%------------
% biblio
%------------
%\newpage
%\bibliographystyle{plainnat}
%\bibliography{seis-flow}
%\nocite{*}
\end{document}