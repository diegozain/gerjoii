\documentclass[11pt, oneside]{article}   	% use "amsart" instead of "article" for AMSLaTeX format
\usepackage{geometry}                		% See geometry.pdf to learn the layout options. There are lots.
\geometry{letterpaper}                   		% ... or a4paper or a5paper or ... 
%\geometry{landscape}                		% Activate for for rotated page geometry
%\usepackage[parfill]{parskip}    		% Activate to begin paragraphs with an empty line rather than an indent
\usepackage{graphicx}				% Use pdf, png, jpg, or eps§ with pdflatex; use eps in DVI mode
								% TeX will automatically convert eps --> pdf in pdflatex		
\usepackage{amssymb}

\title{Code-flow for joint inversion of radar and electrical resistivity}
\author{Diego Domenzain}
\date{}							% Activate to display a given date or no date

\begin{document}
\maketitle
\section{Overview}
This document explains the code-flow for a joint inversion of {\it ground penetrating radar } (GPR) and {\it electrical resistivity} (ER) data using the package {\bf gerjoii}. 
\\\\
The name {\bf gerjoii} stands for {\bf g}round penetrating radar and {\bf e}lectrical {\bf r}esistivity {\bf jo}int {\bf i}nversion and {\bf i}nterferometry. 
\\\\
Our inversion recovers electrical permittivity ($\varepsilon$) and conductivity ($\sigma$) of the subsurface. It assumes a 2.5D subsurface geometry and uses gradient-descent.
\\\\
Job submission is done by shell scripts, and visualization is done with Python. The heavy-lifting code is written in Matlab.
\section{Inside {\it gerjoii}}
The main directory {\bf gerjoii} has sub-directories {\tt data, docs, field} and {\tt src}. 
\\\\
Projects for specific data-sets live in {\tt field}. Inside a particular project in {\tt field} there are sub-directories where inversions with different hyper-parameters were done. 
\\\\
For example, the directory {\tt field/groningen2-fi/g0/} holds the needed scripts for running an inversion on the Groningen data. 
\\\\
Inside each {\tt g\#} there is a directory {\tt scripts}. This directory has Matlab scripts that set-up global variables needed in the inversion. 
\\\\
The next section details what happens next. We go through an inversion-flow whose data claims a very large domain, so we divide it in tiny sub-domains.

\section{Down the rabbit hole - Matlab}
Assume we are in {\tt field/groningen2-fi/g0/scripts/}. The inversion has the following code-flow:

\begin{enumerate}
\item {\bf Run the inversion with {\tt wdc\_begin\_disk\_.m}},
\begin{itemize}
\item load the inversion hyper-parameters, 
\item load the initial $\varepsilon$ and $\sigma$ models,
\item build the tiny sub-domains,
\item load the parallel profile for Matlab,
\item perform the inversion ({\tt wdc\_imageOBJe2\_5d\_disk.m})
\item save the results of the inversion.
\end{itemize}
\item {\bf The inversion {\tt wdc\_imageOBJe2\_5d\_disk.m}}. A while-loop that iterates the following,
\begin{itemize}
\item Compute cross-gradient updates for $\sigma$. Not computationally expensive and not done in parallel.
\item Compute updates for $\sigma$ using the GPR data. Depending on the choice of hyper-parameters, this is done in one of two ways: {\tt w\_update\_s\_disk\_.m} or {\tt w\_updateOBJ\_s\_disk\_.m}. We only review the former,
\begin{itemize}
\item[$\circ$] {\tt w\_update\_s\_disk\_.m} begins a Matlab {\tt parfor} that runs on the number of experiments of the data. The {\tt parfor} runs {\tt w\_update\_s\_disk.m},
\begin{itemize}
\item Loads data on a tiny version of the entire 2D domain. Every step that follows uses just this tiny sub-domain.
\item Solves the GPR forward model, that is Maxwell's 2D wave equation using a finite difference time domain solver. It also saves the entire wavefield (up to 20Gb), a matrix of dimensions length $\times$ depth $\times$ time. This step is CPU and memory intensive. 
\item Solves the GPR forward model using the residual as source, and at each time-step cross-correlates with the recorded wavefield. Once this step is done, the wavefield is deleted. The result of the cross-correlation is the sought update for our gradient-descent. 
\item Finds a step-size for this gradient. The forward model needs to be solved again. It is now an update.
\item Saves this update (which is the size of the tiny domain) in disk.
\end{itemize}
\item[$\circ$] Read all updates that were saved to disk and sum them up to get an update on the entire domain. We now have a GPR update for $\sigma$.
\end{itemize}
\item Compute updates for $\sigma$ using the ER data, {\tt dc\_update2\_5d\_\_.m}.
\begin{itemize}
\item[$\circ$] {\tt dc\_update2\_5d\_\_.m} begins a Matlab {\tt parfor} that runs on the number of experiments of the data. The {\tt parfor} runs {\tt dc\_update2\_5d.m},
\begin{itemize}
\item Loads data in the entire 2D domain.
\item Solves the ER forward model, that is Maxwell's 2.5D steady state equation with {\tt dc\_fwd2\_5d.m},
\begin{itemize}
\item {\tt dc\_fwd2\_5d.m} begins a for-loop of 4 iterations. Each iteration solves and saves in memory a 2D electric potential. These forward models consist of a linear solver and are not CPU nor memory intensive. 
\item Once finished it performs a linear combination of these 2D electric potentials to give one 2.5D electric potential. The 2D electric potentials need to remain saved in memory.
\end{itemize}
\item {\tt dc\_gradient2\_5d.m} computes the 2.5D gradient,
\begin{itemize}
\item {\tt dc\_gradient2\_5d.m} begins a for-loop of 4 iterations. Each iteration solves the ER forward model using the residuals as 2D sources and then multiplying by a special sparse matrix that uses the 2D electric potentials stored in memory.
\item Once finished it performs a linear combination of these 2D gradients to give one 2.5D gradient. The 2D electric potentials are deleted.
\end{itemize}
\item Find a step size for the gradient. This entails solving the 2.5D forward model again for different values of $\sigma$, so all the parts from {\tt dc\_fwd2\_5d.m} are repeated. The gradient is now an update.
\item Save the update in memory. 
\end{itemize}
\item[$\circ$] Sum all updates from all source-receiver pairs together. We now have an ER update for $\sigma$.
\end{itemize}
\item Join $\sigma$ updates from GPR and ER in a special way. This step is not computationally expensive and not done in parallel.
\item Compute cross-gradient updates for $\varepsilon$. Not computationally expensive and not done in parallel.
\item Compute updates for $\varepsilon$ using the GPR data. Depending on the choice of hyper-parameters, this is done in one of two ways: {\tt w\_update\_e\_disk\_.m} or {\tt w\_updateOBJ\_e\_disk\_.m}. These are very similar to those for $\sigma$. The main difference is the number of times the forward model is called when finding the step-size.
\item The while-loop is terminated when the number of iterations wanted is achieved.
\end{itemize}
\end{enumerate}

\section{Notes on time}
Below is a table with approximate timings for the Groningen field data forward models. Each timing is per core per experiment, i.e. one shot-gather (GPR) on one core, or one source-receiver pair (ER) on one core.
\\\\
The times per iteration the forward models are called in the main while-loop ({\tt wdc\_imageOBJe2\_5d\_disk.m}) are also given. We assume we are using {\tt w\_update\_[s,e]\_disk\_.m}. For {\tt w\_updateOBJ\_[s,e]\_disk\_.m} the times the forward models are called is double.
\begin{center}
%\caption{Caption below table.}
\begin{tabular}{| l | c | c | c | c |}
\hline
forward model & time it takes & $\sigma$ update & $\varepsilon$ update & times used\\
\hline
2D    GPR & 20min & \checkmark & &  2\\
2.5D ER &  3min & \checkmark & - &  3 \\
2D    GPR & 20min & & \checkmark &  3\\
\hline
\end{tabular}
\end{center}
The total number of experiments in the GPR and ER data are 56 and 234 respectively.
\\\\
Recall that {\tt w\_update\_s\_disk.m} demands 20Gb per experiment when computing the gradient. This means that out of the 56 experiments we can only do a fraction $f$ at a time, so 20Gb $\times f$ does not exceed memory limits.
\end{document}  










