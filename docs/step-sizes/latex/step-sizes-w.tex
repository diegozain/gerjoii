\documentclass[a4paper,12pt]{article}
%\documentclass[a4paper,10pt]{scrartcl}

\usepackage[utf8x]{inputenc}
\usepackage{amsfonts}
\usepackage{amsmath,esint}
\usepackage{graphicx}
\usepackage{pdfpages}
\usepackage{sansmath}
\usepackage{hyperref}
\usepackage{natbib}
\usepackage{caption}
\usepackage{tikz}
\usepackage{titling}

\setlength{\droptitle}{-10em}   % This is your set screw


\definecolor{boiseBlue} {RGB}{29,72,159}
\definecolor{rojoAmor} {RGB}{171,13,4}
\definecolor{moradoAmor} {RGB}{93,8,113}
\definecolor{verdeAmor} {RGB}{98,158,31}
\definecolor{negro} {RGB}{10,10,10}
\definecolor{lgreen} {RGB}{180,210,100}
\definecolor{dblue}  {RGB}{20,66,129}
\definecolor{ddblue} {RGB}{11,36,69}
\definecolor{lred}   {RGB}{220,0,0}
\definecolor{nred}   {RGB}{224,0,0}
\definecolor{norange}{RGB}{230,120,20}
\definecolor{nyellow}{RGB}{255,221,0}
\definecolor{ngreen} {RGB}{98,158,31}
\definecolor{dgreen} {RGB}{78,138,21}
\definecolor{nblue}  {RGB}{28,130,185}
\definecolor{jblue}  {RGB}{20,50,100}

\usepackage{listings}
\usepackage{xcolor}
\usepackage{verbatim}
\lstset{language=C++,
		basicstyle=\ttfamily,
	       backgroundcolor=\color{black!5}\ttfamily,
                keywordstyle=\color{nblue}\ttfamily,
                stringstyle=\color{nred}\ttfamily,
                commentstyle=\color{ngreen}\ttfamily,
                morecomment=[l][\color{moradoAmor}]{\#}
}

\newenvironment{rcases}{\left.\begin{aligned}}{\end{aligned}\right\rbrace}

\renewcommand{\familydefault}{\sfdefault}

\newcommand{\specialcell}[2][c]{%
  \begin{tabular}[#1]{@{}c@{}}#2\end{tabular}}
% \specialcell{Foo\\bar}

\title{Step sizes for $w$}
\author{}
\date{}

\pdfinfo{%
  /Title    ()
  /Author   ()
  /Creator  ()
  /Producer ()
  /Subject  ()
  /Keywords ()
}

\begin{document}
\maketitle
%-------------------
% main flow
%-------------------
\section*{A tale of units. Theory}
The gradients of the objective function with respect to the parameters computed using the {\it fwi} scheme for electromagnetic waves are,
\begin{align*}
g_{\varepsilon} &= \int_0^T \dot{u}(-t)\cdot a_u(t) \; {\rm d}t, \\
g_{\sigma,w} &= \int_0^T u(-t)\cdot a_u(t) \; {\rm d}t.
\end{align*}
The updates for the parameters will be of the form,
\begin{align*}
\Delta\varepsilon &= -\alpha_\varepsilon \, g_{\varepsilon}, \\
\Delta\sigma &= -\alpha_{\sigma,w} \, g_{\sigma,w},
\end{align*}
and so $\Delta\varepsilon$ and $\Delta\sigma$ must have units of $\varepsilon$ and $\sigma$ respectively ($[F/m],\,[S/m]$). In units, the gradients take the form
\begin{align*}
[g_{\varepsilon}] &= \left[\frac{V}{s\,m}\right] \cdot \left[ \frac{V}{m} \right] \cdot [s]=\frac{m^2\,kg^2}{s^4\,A^2}, \\
[g_{\sigma,w}] &= \left[\frac{V}{m}\right] \cdot \left[ \frac{V}{m} \right] \cdot [s] = \frac{m^2\,kg^2}{s^3\,A^2}.
\end{align*}
By taking the step sizes,
\begin{align*}
[\alpha_{\varepsilon}] &= \frac{1}{A^2} \cdot 
\frac{s^6\,A^6}{m^3\,kg^3} \cdot
\frac{s^2}{m^2} , \\
[\alpha_{\sigma,w}] &= \frac{1}{A^2} \cdot 
\frac{s^6\,A^6}{m^3\,kg^3} \cdot
\frac{1}{m^2},
\end{align*}
we ensure the gradients have the right units. 
\begin{align*}
\left[\frac{F}{m}\right] = \frac{s^4\,A^2}{m^3\,kg}, \hspace{2em} 
\left[\frac{S}{m}\right] = \frac{s^3\,A^2}{m^3\,kg}, \hspace{2em} 
\left[\frac{H}{m}\right] = \frac{m\,kg}{s^2\,A^2}, \hspace{2em}
\left[\frac{V}{m}\right] = \frac{m\,kg}{s^2\,A}.
\end{align*}
\begin{align*}
\varepsilon_o &= 8.8541 \times 10^{-12} & [F/m],\\
\mu_o &= 4\pi \times 10^{-7} & [H/m].
\end{align*}
\section*{A tale of units. Practice}
In practice we compute,
\begin{align*}
g_{\varepsilon} &= \sum_t \dot{u}(-t)\odot a_u(t) \; \Delta t, \\
g_{\sigma,w} &= \sum_t u(-t)\odot a_u(t) \; \Delta t,
\end{align*}
and correct for the right amplitudes by,
\begin{align*}
g_{\varepsilon} &\gets \frac{1}{i^2} \cdot \frac{\Delta t^2}{\varepsilon_o\,\mu_o^3\; \Delta x^2 } \cdot g_{\varepsilon}, \\
g_{\sigma,w} &\gets \frac{1}{i^2} \cdot \frac{1}{\mu_o^3\; \Delta x^2} \cdot g_{\sigma,w}.
\end{align*}
The actual step sizes are computed using Pica,
\begin{align*}
\varepsilon_\bullet &= \varepsilon-k_\varepsilon g_\varepsilon, \\
d_\bullet &= d(\varepsilon,\sigma)-d(\varepsilon_\bullet,\sigma), \\
\alpha_\varepsilon &= k_\varepsilon \frac{d_\bullet(:)^\top \cdot e(:)}{d_\bullet(:)^\top \cdot d_\bullet(:)},
\end{align*}
with a similar equation for $\alpha_\sigma$.
\section*{Finding $k_\varepsilon$}
If $k_\varepsilon$ is too large, then $\varepsilon_\bullet$ falls off the stability region for the wave solver and $\alpha_\varepsilon$ becomes a NaN. If too small, $\alpha_\varepsilon$ is also too small and no relevant update is made on $\varepsilon$.
\\\\
To find the right $k_\varepsilon$, a heuristic search is performed scanning from large to small values of $k_\varepsilon$ and checking if $\varepsilon_\bullet$ falls the stability region or not.
\\\\
The implementation of this search is two while loops, the first loop searches in decreasing order on a coarse discretization of possible values of $k_\varepsilon$. The second loop sweeps a finer discretization of possible values of $k_\varepsilon$ in increasing order.
%\nocite{*}
\end{document}