\documentclass[a4paper,12pt]{article}
%\documentclass[a4paper,10pt]{scrartcl}

\usepackage[utf8x]{inputenc}
\usepackage{amsfonts}
\usepackage{amsmath,esint}
\usepackage{graphicx}
\usepackage{pdfpages}
\usepackage{sansmath}
\usepackage{hyperref}
\usepackage{natbib}

\usepackage{tikz}


\definecolor{boiseBlue} {RGB}{29,72,159}
\definecolor{rojoAmor} {RGB}{171,13,4}
\definecolor{moradoAmor} {RGB}{93,8,113}
\definecolor{verdeAmor} {RGB}{98,158,31}
\definecolor{negro} {RGB}{10,10,10}
\definecolor{lgreen} {RGB}{180,210,100}
\definecolor{dblue}  {RGB}{20,66,129}
\definecolor{ddblue} {RGB}{11,36,69}
\definecolor{lred}   {RGB}{220,0,0}
\definecolor{nred}   {RGB}{224,0,0}
\definecolor{norange}{RGB}{230,120,20}
\definecolor{nyellow}{RGB}{255,221,0}
\definecolor{ngreen} {RGB}{98,158,31}
\definecolor{dgreen} {RGB}{78,138,21}
\definecolor{nblue}  {RGB}{28,130,185}
\definecolor{jblue}  {RGB}{20,50,100}

\usepackage{listings}
\usepackage{xcolor}
\usepackage{verbatim}
\lstset{language=C++,
		basicstyle=\ttfamily,
	       backgroundcolor=\color{black!5}\ttfamily,
                keywordstyle=\color{nblue}\ttfamily,
                stringstyle=\color{nred}\ttfamily,
                commentstyle=\color{ngreen}\ttfamily,
                morecomment=[l][\color{moradoAmor}]{\#}
}

\newenvironment{rcases}{\left.\begin{aligned}}{\end{aligned}\right\rbrace}

\renewcommand{\familydefault}{\sfdefault}

\newcommand{\specialcell}[2][c]{%
  \begin{tabular}[#1]{@{}c@{}}#2\end{tabular}}
% \specialcell{Foo\\bar}

\title{Stochastic gradient descent for GPR and ER}
\author{}
\date{}

\pdfinfo{%
  /Title    ()
  /Author   ()
  /Creator  ()
  /Producer ()
  /Subject  ()
  /Keywords ()
}

\begin{document}
\maketitle
%-------------------
% figures
%-------------------
%
\tiny{
\verbatiminput{../pics/A1.txt}
}
\begin{figure}[t!]
\centering
% left low right up
\includegraphics[trim={120 250 105 295},clip,width=0.5\textwidth]{../pics/A1-s.pdf}~
\includegraphics[trim={120 250 105 295},clip,width=0.5\textwidth]{../pics/A1-s-final-joint.pdf}
\\~\\
\includegraphics[trim={100 250 110 295},clip,width=0.5\textwidth]{../pics/A1-Es-w.pdf}~
\includegraphics[trim={100 250 110 295},clip,width=0.5\textwidth]{../pics/A1-Es-dc.pdf}
\caption{{\bf Top left:} Target conductivity assuming permittivity is known and that it is 2 everywhere. 
{\bf Top right:} Recovered conductivity using joint inversion. 
For GPR, data is measured along {\it all} nodes on the ground-air interface. 
Three sources were evenly spaced along the ground-air interface. 
Regularization is done \`a la Kurzmann (filtering the gradient) with $c_{stab}=1e+1$ and 
with the same ``blanket" taper as for the ER. 
Step-size for the gradient has hyper-parameter $k_\sigma = 1e+0$. 
For the ER, data is measured only on receiver nodes. 
36 source-sink configurations were done. 
Regularization is done by tapering the gradient with a gaussian ``blanket" down to $1\,[m]$. 
Step-size for the gradient has hyper-parameter $k_\sigma = 1e-2$. 
{\bf Bottom:} Objective functions history over iteration. Values are cumulative.
}
\label{fig:fig1}
\end{figure}
%
\begin{figure}[t!]
\centering
% left low right up
\includegraphics[trim={120 250 105 295},clip,width=0.5\textwidth]{../pics/A1-s-large.pdf}~
\includegraphics[trim={120 250 105 295},clip,width=0.5\textwidth]{../pics/A1-s-final-joint-large.pdf}
\\~\\
\includegraphics[trim={100 250 110 295},clip,width=0.5\textwidth]{../pics/A1-Es-w-large.pdf}~
\includegraphics[trim={100 250 110 295},clip,width=0.5\textwidth]{../pics/A1-Es-dc-large.pdf}
\\~\\
\includegraphics[trim={120 250 105 295},clip,width=0.5\textwidth]{../pics/A1-s-final-joint-large-sicodelica.pdf}
\caption{{\bf Top left:} Target conductivity assuming permittivity is known and that it is 2 everywhere. 
{\bf Top right:} Recovered conductivity using joint inversion. 
For GPR, data is measured along {\it all} nodes on the ground-air interface. 
Three sources were evenly spaced along the ground-air interface. 
Regularization is done \`a la Kurzmann (filtering the gradient) with $c_{stab}=1e+1$ and 
with the same ``blanket" taper as for the ER. 
Step-size for the gradient has hyper-parameter $k_\sigma = 1e+0$. 
For the ER, data is measured only on receiver nodes. 
36 source-sink configurations were done. 
Regularization is done by tapering the gradient with a gaussian ``blanket" down to $1\,[m]$. 
Step-size for the gradient has hyper-parameter $k_\sigma = 1e-2$. 
{\bf Middle:} Objective functions history over iteration. Values are cumulative. 
{\bf Bottom:} Psycodelic picture of recovered conductivity.
}
\label{fig:fig2}
\end{figure}
%
\begin{figure}[t!]
\centering
% left low right up
\includegraphics[trim={120 250 105 295},clip,width=0.5\textwidth]{../pics/A1-s-large.pdf}
\\~\\
\includegraphics[trim={100 250 105 295},clip,width=0.5\textwidth]{../pics/A1-s-final-w-large.pdf}~
\includegraphics[trim={100 250 110 295},clip,width=0.5\textwidth]{../pics/A1-Es-w-proper-large.pdf}
\\~\\
\includegraphics[trim={100 250 105 295},clip,width=0.5\textwidth]{../pics/A1-s-final-dc-large.pdf}~
\includegraphics[trim={100 250 110 295},clip,width=0.5\textwidth]{../pics/A1-Es-dc-proper-large.pdf}
\caption{{\bf Top:} Target conductivity assuming permittivity is known and that it is 2 everywhere. 
{\bf Middle:} GPR inversion. Data is measured along {\it all} nodes on the ground-air interface. 
Three sources were evenly spaced along the ground-air interface. 
Regularization is done \`a la Kurzmann (filtering the gradient) with $c_{stab}=1e+1$ and 
with the same ``blanket" taper as for the ER. 
Step-size for the gradient has hyper-parameter $k_\sigma = 1e+0$. 
{\bf Bottom:} ER inversion. Data is measured only on receiver nodes. 
36 source-sink configurations were done. 
Regularization is done by tapering the gradient with a gaussian ``blanket" down to $1\,[m]$. 
Step-size for the gradient has hyper-parameter $k_\sigma = 1e-2$. 
}
\label{fig:fig3}
\end{figure}
%
\begin{figure}[t!]
\centering
% left low right up
\includegraphics[trim={120 300 105 320},clip,width=0.5\textwidth]{../pics/A1-s-large.pdf}~
\includegraphics[trim={120 300 105 320},clip,width=0.5\textwidth]{../pics/A1-s-final-joint-large-1323.pdf}
\\~\\
\includegraphics[trim={100 300 110 320},clip,width=0.5\textwidth]{../pics/A1-Es-w-large-1323.pdf}~
\includegraphics[trim={100 300 110 320},clip,width=0.5\textwidth]{../pics/A1-Es-dc-large-1323.pdf}
\\~\\
\includegraphics[trim={100 300 110 320},clip,width=0.5\textwidth]{../pics/A1-step-w-1323.pdf}~
\includegraphics[trim={100 300 110 320},clip,width=0.5\textwidth]{../pics/A1-step-dc-1323.pdf}
\\~\\
\includegraphics[trim={110 300 105 320},clip,width=0.5\textwidth]{../pics/A1-s-final-joint-large-1323-clipped.pdf}~
\includegraphics[trim={110 300 105 320},clip,width=0.5\textwidth]{../pics/A1-s-final-joint-large-sicodelica-1323-clipped.pdf}
\caption{{\bf Top left:} Target conductivity assuming permittivity is known and that it is 2 everywhere. 
{\bf Top right:} Recovered conductivity using joint inversion. 
Hyper-parameters the same as Figure (\ref{fig:fig2}).  
{\bf Middle top:} History of objective functions over iterations. 
{\bf Middle bottom:} History of step sizes for gradient updates over iterations. 
{\bf Bottom:} Pictures of recovered conductivity clipped to true values.
}
\label{fig:fig2}
\end{figure}
%\verbatiminput{../gd-stoch/A1-reality-s.txt}
%
%------------
% biblio
%------------
%\newpage
%\bibliographystyle{plainnat}
%\bibliography{seis-flow}
%\nocite{*}
\end{document}